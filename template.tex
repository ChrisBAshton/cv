%%%%%%%%%%%%%%%%%
% This CV theme is mmayer.tex
%%%%%%%%%%%%%%%%%

%% If you are using \orcid or academicons
%% icons, make sure you have the academicons
%% option here, and compile with XeLaTeX
%% or LuaLaTeX.
% \documentclass[10pt,a4paper,academicons]{altacv}
%%
%% Use the "normalphoto" option if you want a normal photo instead of cropped to a circle
% \documentclass[10pt,a4paper,normalphoto]{altacv}
\documentclass[10pt,a4paper,ragged2e,withhyper]{altacv}
% PUBLIC OR PRIVATE
\def\customcontact{Website: \url{http://ashton.codes}; Email: chris (AT) ashton.codes}
\def\customreferences{References available upon request.}
\documentclass[class=article, crop=false]{standalone}

%--------------------BEGIN DOCUMENT----------------------
\begin{document}

%--------------------TITLE-------------
\par{\centering
        {\Huge Christopher Benjamin \textsc{Ashton}
    }\bigskip\par}

\begin{center}
\customcontact
\end{center}

%--------------------SECTIONS-----------------------------------

\section{Profile}

Full-stack web developer and award-winning 1\textsuperscript{st} class Software Engineering graduate (\emph{Best Bachelor Degree in Computer Science, Best Major Project}); I am a highly efficient, organised and creative individual currently working full-time for the BBC. In my spare time I run my own digital agency and am a tenor in the BBC Symphony Chorus.

\section{Key Achievements and Skills}

\begin{itemize}
    \setlength\itemsep{0.3em}

    \item \textbf{Technical leadership} –  responsible for the technical roadmap within BBC News' Visual Journalism team, recently leading the team's upgrade to a modular ES6 architecture. Collaborate across teams to devise maintainable, future-proof solutions, demonstrated by working with the News App team to agree on an API for consumption of Visual Journalism content.

    \item \textbf{Communication skills} - led BBC Visual Journalism's adoption of user stories as a communication tool and a developer aid. Regularly give well-received presentations to technical and non-technical audiences at all levels. Work effectively in scrum teams adopting agile techniques.

    \item \textbf{Automated testing} – lead maintainer of the Wraith visual regression testing tool, which has gone on to become the BBC's most-starred open-source project. Highly adept at authoring declarative Cucumber features alongside clean, reusable Ruby step definitions. Hand-built an in-house BDD framework in Node. Set up suites of automated tests in Docker on Jenkins CI.

    \item \textbf{Manual testing} – tested apps across multiple devices and browsers in my role as Test Engineer at the BBC, using exploratory techniques to validate application behaviour and responsiveness. Devised and documented an efficient manual-testing strategy, dubbed the ``Three Phase Attack'', in a well-received article published on \emph{Smashing Magazine}.

    \item \textbf{Accessibility} - currently volunteer as an Accessibility Champion at the BBC, promoting accessible web design and development within my team. Write semantic HTML5 augmented with accessibility attributes, and have experience of using schemas for structured data. Tested sites against WAI validators, colour contrast checkers, and screen readers in my role as Test Engineer.

    \item \textbf{SASS, CSS3 \& BEM} - I was the main developer of the Commonwealth Games Quiz (\emph{Best News Data App}, Data Journalism Awards 2015), featured on BBC Breakfast and most read and shared article on BBC News on the day of publication: \url{http://bbc.in/1ll3XHE}. Developed to support internationalisation, I took the initiative to translate it into Welsh: \url{http://bbc.in/1nDNxjD}

    \item \textbf{JavaScript \& Node} - expert developer of self-documenting, object-oriented, unobtrusive JavaScript developed through TDD (Mocha/Chai). Equally comfortable whether writing NPM modules in ES6 or cross-browser ES5 backwards-compatible to IE6. Developed a 3D solar system in WebGL: \url{http://chrisbashton.github.io/solar-system-webgl/source/}

    \item \textbf{Back-end web development} - highly experienced with OO PHP and MVCR architecture, using frameworks such as Zend and F3. Apply techniques such as PDO to prevent SQL injection attacks, encryption and salting for sensitive data, and have knowledge of preventing XSS attacks.

    \item \textbf{AWS} - developed a horizontally scalable, hosted webhooks platform: \url{http://githooks.io}. This side project runs on LAMP architecture on an EC2 instance, and uses a clever method for securely passing GitHub access tokens to webhooks which are executed on AWS Lambda.

    \item \textbf{WordPress} – created a number of bespoke themes and plugins, including plugins defining custom taxonomies (`app reviews'): \url{http://musicmakerapps.com}; a responsive e-commerce theme: \url{http://hover-bike.com}; and parent/child themes for related magazine websites: \linebreak\url{http://loopinglive.com} and \url{http://thevintagemusician.com}.

\end{itemize}

\section{Employment}
\begin{tabular}{r|p{11cm}}
%------------------------------
\textsc{02/2016 - Present}
&Web Developer
\\\emph{BBC, London}
&\footnotesize{
\par{Led the team's upgrade to a modular ES6 infrastructure and continue to be responsible for the team's technical roadmap.}
\par{Regularly create engaging, full-stack applications in up to 28 languages online and for the BBC News and Sport apps. These are developed for high-profile events with immovable deadlines, such as the Olympic Games, EU Referendum and US Election.}
\par{Helped to recruit for junior roles by screening CVs and participating in interviews and coding exercises. Act as first line of support for junior colleagues, with increasing line management duties.}
}
%------------------------------
\\\multicolumn{2}{c}{} \\
\textsc{07/2015 - 01/2016}
&Test Engineer
\\\emph{BBC, London}
&\footnotesize{
Won the \emph{Best Contributor of the Year} award for my contribution to Wraith. Set up suites of automated Cucumber and Wraith tests (using Docker) on CI.  Devised the cost-effective ``Three Phase Attack'' testing strategy, published in Smashing Magazine:
\url{https://www.smashingmagazine.com/2016/02/high-impact-minimal-effort-cross-browser-testing/}
}
%------------------------------
\\\multicolumn{2}{c}{} \\
\textsc{08/2014 - Present}
&Founder
\\\emph{Webdapper Ltd}
&\footnotesize{Led the development of over 10 websites, continuing to maintain them through project-managing a team of trusted developers and designers. Generated over 20,000 Facebook likes for my clients, through carefully planned advertizing campaigns.}
%------------------------------
\\\multicolumn{2}{c}{} \\
\textsc{07/2013 - 08/2014}
&Trainee Web Developer
\\\emph{BBC, London}
&\footnotesize{Created `mobile-first', bespoke applications to tight deadlines for the BBC News website, delivering to an audience of millions. Built a `Common Tasks' toolkit to ease development workflows, cutting the time taken to perform some tasks down from half an hour to under a minute. Templated the ‘datapic’ format: \url{http://bit.ly/1wT6XTz}}
\end{tabular}

\section{Education}

\begin{tabular}{rp{11cm}}
\textsc{2011 - 2015}
&
\textbf{Aberystwyth University}, Wales
\\& BEng (Hons) Software Engineering (1\textsuperscript{st}). BCS-accredited. Grade: 84\%

Dissertation: \emph{Online Dispute Resolution for Maritime Collisions}: an open-source ODR platform with plugin support for AI-driven automated resolutions. Hosted on AWS: \url{http://smartresolution.org}

\\
\textsc{2003 - 2010}
&
\textbf{Ysgol Gyfun Llanbedr Pont Steffan}, Wales
\\& \begin{tabular}{lp{11cm}}
    A Levels: & Mathematics (A), Music (A), Chemistry (B)
    \\AS Level: & Biology (B)
    \\GCSEs: & 13 total, 11 at grade A*/A
    \end{tabular}

\end{tabular}

\section{Awards and Affiliations}
\begin{tabular}{lp{11cm}}
\emph{Best Contributor of the Year}
&
BBC News Developer Awards, 2015
\\
\emph{Best Bachelor Degree in Computer Science}
&
Aberystwyth University, 2015
\\
\emph{British Computer Society Prize for the Best Major Project}
&
Aberystwyth University, 2015
\end{tabular}

\section{Interests}
I'm a tenor in the \emph{BBC Symphony Chorus} and regularly perform complex choral repertoire at internationally televised concerts. Previously: \emph{National Youth Choir of Wales} (Tenor: two years), \emph{Elizabethan Madrigal Singers} (Choral Director: one year, Tenor: two years).

\section{References}
\customreferences

\end{document}

%% AltaCV uses the fontawesome5 and academicon fonts
%% and packages.
%% See http://texdoc.net/pkg/fontawesome5 and http://texdoc.net/pkg/academicons for full list of symbols. You MUST compile with XeLaTeX or LuaLaTeX if you want to use academicons.

% Change the page layout if you need to
\geometry{left=1.25cm,right=1.25cm,top=1.5cm,bottom=1.5cm,columnsep=1.2cm}

% The paracol package lets you typeset columns of text in parallel
\usepackage{paracol}

% For configuring nested lists
\usepackage{enumitem}

% Change the font if you want to, depending on whether
% you're using pdflatex or xelatex/lualatex
\ifxetexorluatex
  % If using xelatex or lualatex:
  \setmainfont{Lato}
\else
  % If using pdflatex:
  \usepackage[default]{lato}
\fi

% Change the colours if you want to
\definecolor{VividPurple}{HTML}{3E0097}
\definecolor{SlateGrey}{HTML}{2E2E2E}
\definecolor{LightGrey}{HTML}{666666}
\definecolor{LinkBlue}{HTML}{0B0080}
% \colorlet{name}{black}
\colorlet{tagline}{VividPurple}
\colorlet{heading}{VividPurple}
\colorlet{headingrule}{VividPurple}
% \colorlet{subheading}{PastelRed}
\colorlet{accent}{VividPurple}
\colorlet{customlinkcolor}{LinkBlue}
\colorlet{emphasis}{SlateGrey}
\colorlet{body}{LightGrey}

% Change some fonts, if necessary
% \renewcommand{\namefont}{\Huge\rmfamily\bfseries}
% \renewcommand{\personalinfofont}{\footnotesize}
% \renewcommand{\cvsectionfont}{\LARGE\rmfamily\bfseries}
% \renewcommand{\cvsubsectionfont}{\large\bfseries}

% Change the bullets for itemize and rating marker
% for \cvskill if you want to
\renewcommand{\itemmarker}{{\small\textbullet}}
\renewcommand{\ratingmarker}{\faCircle}

\addbibresource{publications.bib}
% PREVENT "n.d." appearing when we don't provide date
% in publications.bib entries. See:
% https://tex.stackexchange.com/questions/336041/remove-n-d-no-date-from-online-entries-without-dates-in-biblatex
\DeclareLabeldate{%
  \field{date}
  \field{year}
  \field{eventdate}
  \field{origdate}
  \field{urldate}
}

\begin{document}
\name{Chris Ashton}
\tagline{Full-stack Software Engineer}
% Cropped to square from https://en.wikipedia.org/wiki/Marissa_Mayer#/media/File:Marissa_Mayer_May_2014_(cropped).jpg, CC-BY 2.0
%% You can add multiple photos on the left or right
% \photoR{2.5cm}{mmayer-wikipedia-cc-by-2_0}
% \photoL{2cm}{Yacht_High,Suitcase_High}
\personalinfo{%
  % Not all of these are required!
  % You can add your own with \printinfo{symbol}{detail}
  \email{\cvmail}
  \phone{\cvnumberphone}
%   \mailaddress{London}
  \location{Cambridge, UK}
  \homepage{ashton.codes}
  \github{ChrisBAshton}
%  \twitter{@ChrisBAshton}
%  \linkedin{chrisbashton}
%   \orcid{orcid.org/0000-0000-0000-0000} % Obviously making this up too. If you want to use this field (and also other academicons symbols), add "academicons" option to \documentclass{altacv}
  %% You MUST add the academicons option to \documentclass, then compile with LuaLaTeX or XeLaTeX, if you want to use \orcid or other academicons commands.
  % \orcid{0000-0000-0000-0000}
  %% You can add your own arbtrary detail with
  %% \printinfo{symbol}{detail}[optional hyperlink prefix]
  % \printinfo{\faPaw}{Hey ho!}
  %% Or you can declare your own field with
  %% \NewInfoFiled{fieldname}{symbol}[optional hyperlink prefix] and use it:
  % \NewInfoField{gitlab}{\faGitlab}[https://gitlab.com/]
  % \gitlab{your_id}
}

\makecvheader

%% Depending on your tastes, you may want to make fonts of itemize environments slightly smaller
\AtBeginEnvironment{itemize}{\small}

%% Set the left/right column width ratio to 6:4.
\columnratio{0.6}

% Start a 2-column paracol. Both the left and right columns will automatically
% break across pages if things get too long.
\begin{paracol}{2}

\def\nestedlistspacer{0.1cm}

\cvsection{Experience}

%\cvevent{Developer}{Government Digital Service}{April 2019 -- Present}{LOCATION GOES HERE IF YOU WANT}

\cvevent{Lead Developer}{Government Digital Service}{July 2023 -- Present}{}

\begin{itemize}
    \item Lead Developer in the platform space, on 6 month EOI.
\end{itemize}

\divider

\cvevent{Senior Developer}{}{June 2021 -- July 2023}{}

\begin{itemize}
    \item \textbf{Technical Lead on the Platform Security \& Reliability team.}

    \vspace{\nestedlistspacer}
    \begin{itemize}[label=$\triangleright$]
        \item Work closely with Delivery Manager and Product Owner to plan and prioritise work for our team of developers, SREs and security experts.
        \item Regularly share knowledge through show and tells, mobbing sessions, code reviews and pairing. Introduced the 'mission leads' model, giving team members opportunities to gain leadership experience.
        \item Wrote and delivered RFC-156 for automatic merging of internal Dependabot PRs, eliminating a toil task and freeing up developer time. Required considerable technical planning to build in a secure manner.
        \item Reviewed and implemented the design for the new postcode lookup solution on GOV.UK, in a way that can scale with spikes in traffic.
    \end{itemize}

    \item \textbf{Technical Lead on the 2nd Line Support function}.

    \vspace{\nestedlistspacer}
    \begin{itemize}[label=$\triangleright$]
        \item Led efforts to broaden our drilling strategy to better prepare engineers for incidents. Rewrote our incident guidance documentation.
        \item Created a rota generator to more fairly allocate support shifts, and drove change in shift pattern to increase available pool of developers.
        \item Drove a change in policy to reduce the level of access we grant apprentices. Rewrote our user monitoring systems to make access more explicit \& predictable. Automated much of this with Terraform.
    \end{itemize}

    \item Led GOV.UK's response to a security advisory, liaising with tech leads and overseeing the patching of all vulnerable apps within a few days. Later raised RFC-153 to enable automated security PRs.

    \item Independently discovered and patched a cross-site scripting bug on a GOV.UK subdomain. I was credited in the security advisory. \href{https://github.com/alphagov/tech-docs-gem/security/advisories/GHSA-x2xw-hw8g-6773}{\url{github.com/alphagov/tech-docs-gem/security/advisories/GHSA-x2xw-hw8g-6773}}

    \item I was GOV.UK's point of contact for a penetration test conducted by the cyber team, requiring walking them through the brief, explaining the architecture of GOV.UK and ensuring they had the necessary access.
\end{itemize}

\divider

\cvevent{Developer}{}{April 2019 -- May 2021}{}
\begin{itemize}
    \item Technical Lead of the Developer Tools community. Made significant improvements to the processes around our documentation.
    % 70k-150k weekly Sentry errors in early-mid 2020, reduced to just 8.5k in mid January 2021.
    \item Led a stream of work to reduce application errors across GOV.UK by over 90\%. Introduced alerting so that new issues can be quickly fixed.
    \item Simplified infrastructure and cut costs by migrating a static site from S3 to GitHub Pages, and its build job from Jenkins to GitHub Actions.
    \item Contributed to a number of Ruby on Rails applications, working in agile, multidisciplinary teams that use scrum and kanban, adding new features backed by user research and fixing bugs through TDD.
    \item I've led on incidents, using alerts, dashboards and logs to debug issues on production. I've used DevOps techniques to monitor disk space and processes on machines, and used ActiveRecord and SQL queries to search PostgreSQL and MongoDB records.
    \item Proactively paired with new developers on my team, and was designated the 'buddy' to a new joiner. Started line-managing an apprentice.
    \item Co-ordinated an audit of all GOV.UK repositories, archiving 27 legacy repos to reduce the overhead of keeping dependencies up to date.
    \item Sifted CVs and took part in phone interviews for several developer roles. Wrote article about working at GDS, subsequently used for recruitment.
\end{itemize}

\divider

\cvevent{Acting Technical Lead}{BBC}{Nov 2018 -- April 2019}{}
\begin{itemize}
    \item Led the delivery of v1 of the new BBC article renderer: an isomorphic React single page application and shared component library.
\end{itemize}

\divider

\cvevent{Senior Software Engineer}{}{Feb 2017 -- Oct 2018}{}

\begin{itemize}
    \item Recruited and trained developers in India and Indonesia as part of BBC's global expansion. Line managed and mentored around a dozen developers, including several from the trainee and graduate schemes.
    \item Upgraded legacy tech stack to pure Node/NPM and Webpack (including bespoke plugins). Migrated deployment pipeline to AWS.
    \item Volunteered as ``Accessibility Champion'' and improved standards within the department, such as including progressive enhancement by default.
\end{itemize}

\divider

\cvevent{Web Developer}{}{Feb 2016 -- Jan 2017}{}

\begin{itemize}
    \item Devised and implemented an abstraction strategy to deliver Visual Journalism content across the BBC apps, web and third-parties. Wrote about this on \emph{Smashing Magazine} and spoke about it at \emph{Paris Web Conf}.
    \item Regularly created multi-lingual, full-stack applications for the web and the BBC apps, covering high profile events to tight deadlines.
\end{itemize}

\divider

\cvevent{Test Engineer}{}{July 2015 -- Jan 2016}{}

\begin{itemize}
    \item Led the maintenance of a screenshot testing tool (Wraith). Hand-built a BDD framework in Node. Set up Docker test suites on CI.
    \item Awarded \emph{Best Contributor of the Year} at the BBC News Developer Awards.
\end{itemize}

\divider

\cvevent{Trainee Web Developer}{}{July 2013 -- Aug 2014}{}

\begin{itemize}
    \item Automated a large number of developer tasks, dramatically improving team productivity. Working with a senior designer, I templated visual formats that previously had to be crafted from scratch.
    \item Developed the \emph{Commonwealth Games} feature, which was televised on BBC Breakfast. Being bilingual, I also took the initiative to translate it to Welsh. \url{https://www.bbc.co.uk/news/uk-28062001}
\end{itemize}

\divider

\cvevent{Founder}{Webdapper Ltd}{Aug 2014 -- Present}{}

\begin{itemize}
    \item Continued contracting for the BBC (one day per week) through Webdapper, which I founded during my final year at university. To date, I've also planned, designed and developed over 10 websites for other clients.
\end{itemize}

\divider

%% Switch to the right column. This will now automatically move to the second
%% page if the content is too long.
\switchcolumn

\cvsection{Summary}
\begin{quote}
I am a creative, organised, pragmatic
\\
web developer with extensive back end
\\
and front end experience.
\end{quote}

% \cvsection{Achievements}

% \cvachievement{\faTrophy}{BBC Symphony Chorus (2013 - 2023)}{Have sung in several televised Proms}

\cvsection{Education}

\cvevent{B.Eng. in Software Engineering \newline 1st class, with Honours, BCS-accredited}{Aberystwyth University, Wales}{Sept 2011 -- June 2015}{}

\begin{itemize}
    \item \textbf{Graduated top of the class} and awarded \emph{Best Bachelor Degree in Computer Science} out of approximately 80 people.
    \item Technologies used: Java, C++, C, Haskell, Python, Ruby, PHP.
    \item Key modules: Advanced Computer Graphics, Agile Methodologies, Machine Learning, Computer Vision, Data Structures \& Algorithms.
\end{itemize}

\divider

\cvevent{AAB (A-Levels), B (AS Level), \newline 13 GCSEs (11 at A*/A)}{Lampeter Comprehensive School, Wales}{}{}

\cvsection{Publications}

\def\publicationspacer{0.2cm}

\cvsubsection{\printinfo{\faBook}{Newsletters}}
\textbf{frequent11y} (2019 - Present)
\newline
I've published almost 1000 a11y newsletters to a large cross-government community and hundreds of independent subscribers.
\newline
\href{https://ashton.codes/subscribe-to-frequent11y}{\url{https://ashton.codes/subscribe-to-frequent11y}}

\vspace{\publicationspacer}
\cvsubsection{\printinfo{\faUsers}{Talks}}
\textbf{Paris Web conference}
\newline
I represented the BBC in 2018 with my talk, "Publishing content to multiple platforms".
\newline
\href{https://vimeo.com/295213010}{\url{https://vimeo.com/295213010}}

\vspace{\publicationspacer}
\cvsubsection{\printinfo{\faFile*[regular]}{Articles}}
\begin{itemize}
    \item \textbf{What I like about being a developer at GDS} \newline \href{https://technology.blog.gov.uk/2019/08/14/what-i-like-about-being-a-developer-at-gds/}{\url{technology.blog.gov.uk/2019/08/14/what-i-like-about-being-a-developer-at-gds/}}

    \item \textbf{I Used The Web For A Day...}
    \newline Series of 5 accessibility articles for \emph{Smashing Magazine}.
    \newline
    \href{https://www.smashingmagazine.com/author/chrisbashton/}{\url{smashingmagazine.com/author/chrisbashton}}
\end{itemize}

\newpage

\cvsection{Projects}

\def\spacer{0.7cm}

\textbf{StudentMunch.com} - food review site, using structured data markup to show as reviews in search results. Built as a Progressive Web App on top of AMP.

\medskip

  \cvtag{Service Workers}
  \cvtag{AMP}
  \cvtag{Offline first}
  \\
  \cvtag{PWA}
  \cvtag{PRPL pattern}
  \cvtag{Structured data}

\vspace{\spacer}

\textbf{peepshowquot.es} - fan website for Channel 4 sitcom \emph{Peep Show}: a searchable directory of gifs for every quote.

\medskip

  \cvtag{React}
  \cvtag{NodeJS}
  \cvtag{Jamstack}
  \\
  \cvtag{AWS Lambda}
  \cvtag{Serverless}

\vspace{\spacer}

\textbf{SmartResolution} - online dispute resolution platform with plugin support for AI-driven resolutions. Awarded the Best Major Project prize at university.
\newline
\href{https://github.com/ChrisBAshton/smartresolution}{\url{github.com/ChrisBAshton/smartresolution}}

\medskip

  \cvtag{Cucumber \& Ruby}
  \cvtag{PHP}
  \cvtag{Composer}
  \cvtag{SQL}
  \cvtag{PDO}
  \cvtag{LAMP}

\vspace{\spacer}

\textbf{Secretary} - automatic quality assurance editorial checks for WordPress posts. Built to be extensible and config-driven.
\newline
\href{https://wordpress.org/plugins/secretary/}{\url{wordpress.org/plugins/secretary/}}

\medskip

  \cvtag{WordPress}
  \cvtag{YAML}
  \cvtag{SVN}
  \cvtag{Open-source}

\medskip

\textbf{3D Solar System} - solar system written in WebGL.
\newline
\href{https://chrisbashton.github.io}{\url{chrisbashton.github.io}}

\medskip

  \cvtag{WebGL}
  \cvtag{JavaScript}
  \cvtag{HTML5}
  \cvtag{CSS3}

\medskip

%\cvsection{Languages}
%\cvskill{English}{5}
%\cvskill{Welsh}{3}
%\newpage

\cvsection{Referees}

\customreferences

\end{paracol}

\end{document}
