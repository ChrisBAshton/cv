%%%%%%%%%%%%%%%%%
% This CV theme is mmayer.tex
%%%%%%%%%%%%%%%%%

%% If you are using \orcid or academicons
%% icons, make sure you have the academicons
%% option here, and compile with XeLaTeX
%% or LuaLaTeX.
% \documentclass[10pt,a4paper,academicons]{altacv}
%%
%% Use the "normalphoto" option if you want a normal photo instead of cropped to a circle
% \documentclass[10pt,a4paper,normalphoto]{altacv}
\documentclass[10pt,a4paper,ragged2e,withhyper]{altacv}
% PUBLIC OR PRIVATE
\input{cv--private}

%% AltaCV uses the fontawesome5 and academicon fonts
%% and packages.
%% See http://texdoc.net/pkg/fontawesome5 and http://texdoc.net/pkg/academicons for full list of symbols. You MUST compile with XeLaTeX or LuaLaTeX if you want to use academicons.

% Change the page layout if you need to
\geometry{left=1.25cm,right=1.25cm,top=1.5cm,bottom=1.5cm,columnsep=1.2cm}

% The paracol package lets you typeset columns of text in parallel
\usepackage{paracol}

% Change the font if you want to, depending on whether
% you're using pdflatex or xelatex/lualatex
\ifxetexorluatex
  % If using xelatex or lualatex:
  \setmainfont{Lato}
\else
  % If using pdflatex:
  \usepackage[default]{lato}
\fi

% Change the colours if you want to
\definecolor{VividPurple}{HTML}{3E0097}
\definecolor{SlateGrey}{HTML}{2E2E2E}
\definecolor{LightGrey}{HTML}{666666}
\definecolor{LinkBlue}{HTML}{0B0080}
% \colorlet{name}{black}
\colorlet{tagline}{VividPurple}
\colorlet{heading}{VividPurple}
\colorlet{headingrule}{VividPurple}
% \colorlet{subheading}{PastelRed}
\colorlet{accent}{VividPurple}
\colorlet{customlinkcolor}{LinkBlue}
\colorlet{emphasis}{SlateGrey}
\colorlet{body}{LightGrey}

% Change some fonts, if necessary
% \renewcommand{\namefont}{\Huge\rmfamily\bfseries}
% \renewcommand{\personalinfofont}{\footnotesize}
% \renewcommand{\cvsectionfont}{\LARGE\rmfamily\bfseries}
% \renewcommand{\cvsubsectionfont}{\large\bfseries}

% Change the bullets for itemize and rating marker
% for \cvskill if you want to
\renewcommand{\itemmarker}{{\small\textbullet}}
\renewcommand{\ratingmarker}{\faCircle}

\addbibresource{publications.bib}
% PREVENT "n.d." appearing when we don't provide date
% in publications.bib entries. See:
% https://tex.stackexchange.com/questions/336041/remove-n-d-no-date-from-online-entries-without-dates-in-biblatex
\DeclareLabeldate{%
  \field{date}
  \field{year}
  \field{eventdate}
  \field{origdate}
  \field{urldate}
}

\begin{document}
\name{Chris Ashton}
\tagline{Full-stack Software Engineer}
% Cropped to square from https://en.wikipedia.org/wiki/Marissa_Mayer#/media/File:Marissa_Mayer_May_2014_(cropped).jpg, CC-BY 2.0
%% You can add multiple photos on the left or right
% \photoR{2.5cm}{mmayer-wikipedia-cc-by-2_0}
% \photoL{2cm}{Yacht_High,Suitcase_High}
\personalinfo{%
  % Not all of these are required!
  % You can add your own with \printinfo{symbol}{detail}
  \email{\cvmail}
  \phone{\cvnumberphone}
%   \mailaddress{London}
  \location{Cambridge, UK}
  \homepage{ashton.codes}
  \github{ChrisBAshton}
%  \twitter{@ChrisBAshton}
%  \linkedin{chrisbashton}
%   \orcid{orcid.org/0000-0000-0000-0000} % Obviously making this up too. If you want to use this field (and also other academicons symbols), add "academicons" option to \documentclass{altacv}
  %% You MUST add the academicons option to \documentclass, then compile with LuaLaTeX or XeLaTeX, if you want to use \orcid or other academicons commands.
  % \orcid{0000-0000-0000-0000}
  %% You can add your own arbtrary detail with
  %% \printinfo{symbol}{detail}[optional hyperlink prefix]
  % \printinfo{\faPaw}{Hey ho!}
  %% Or you can declare your own field with
  %% \NewInfoFiled{fieldname}{symbol}[optional hyperlink prefix] and use it:
  % \NewInfoField{gitlab}{\faGitlab}[https://gitlab.com/]
  % \gitlab{your_id}
}

\makecvheader

%% Depending on your tastes, you may want to make fonts of itemize environments slightly smaller
\AtBeginEnvironment{itemize}{\small}

%% Set the left/right column width ratio to 6:4.
\columnratio{0.6}

% Start a 2-column paracol. Both the left and right columns will automatically
% break across pages if things get too long.
\begin{paracol}{2}

\cvsection{Experience}

%\cvevent{Developer}{Government Digital Service}{April 2019 -- Present}{LOCATION GOES HERE IF YOU WANT}
\cvevent{Developer}{Government Digital Service}{April 2019 -- Present}{}
\begin{itemize}
    % 70k-150k weekly Sentry errors in early-mid 2020, reduced to just 8.5k in mid January 2021.
    \item Led a stream of work to reduce application errors across GOV.UK by over 90\%. Introduced alerting so that new issues can be quickly fixed.
    \item Currently leading an informal team responsible for rolling out Continuous Deployment and improving developer tooling.
    \item Regularly work on a number of Ruby on Rails applications: fixing bugs through TDD, upgrading dependencies, adding new features backed by user research, and contributing to GOV.UK Design System components.
    \item Use Icinga alerts, Grafana dashboards, Kibana logs and Sentry errors to debug issues on production. I've lead on incidents, using DevOps techniques to monitor disk space and processes on machines, and Rails console to query PostgreSQL / MongoDB records with ActiveRecord.
    \item Work in agile, multidisciplinary teams that adopt a mixture of scrum and kanban, and have worked with stakeholders to write Trello cards.
    \item Simplified infrastructure and cut costs by migrating a static site from S3 to GitHub Pages, and its build job from Jenkins to GitHub Actions.
    \item Proactively paired with new developers on my team, and was designated the 'buddy' to a new joiner. I now line manage an apprentice developer.
    \item Co-ordinated an audit of all GOV.UK repositories, archiving 27 legacy repos to reduce the overhead of keeping dependencies up to date.
    \item Sifted CVs and took part in phone interviews for several developer roles. Wrote article about working at GDS, subsequently used for recruitment.
\end{itemize}

\divider

\cvevent{Acting Technical Lead}{BBC, Articles}{Nov 2018 -- April 2019}{}
\begin{itemize}
    \item Led the delivery of v1 of the new BBC article renderer: an isomorphic React single page application and shared component library.
\end{itemize}

\divider

\cvevent{Senior Software Engineer}{BBC, Visual Journalism}{Feb 2017 -- Oct 2018}{}

\begin{itemize}
    \item Recruited and trained a global team. Line managed and mentored several developers, allocating projects aligning with their goals.
    \item Upgraded legacy tech stack to pure Node/NPM and Webpack (including bespoke plugins). Migrated deployment pipeline to AWS.
    \item Volunteered as ``Accessibility Champion'' and improved standards.
\end{itemize}

\divider

\cvevent{Web Developer}{BBC, Visual Journalism}{Feb 2016 -- Jan 2017}{}

\begin{itemize}
    \item Devised and implemented an abstraction strategy to deliver Visual Journalism content across the BBC apps, web and third-parties. Wrote about this on \emph{Smashing Magazine} and spoke about it at \emph{Paris Web Conf}.
    \item Regularly created multi-lingual, full-stack applications for the web and the BBC apps, covering high profile events to tight deadlines.
\end{itemize}

% UNCOMMENT THIS IF BELOW JOB APPEARS ON PAGE 1
% \divider

\cvevent{Test Engineer}{BBC, Visual Journalism}{July 2015 -- Jan 2016}{}

\begin{itemize}
    \item Led the maintenance of a screenshot testing tool (Wraith). Hand-built a BDD framework in Node. Set up Docker test suites on CI.
\end{itemize}

\divider

\cvevent{Founder}{Webdapper Ltd}{Aug 2014 -- Present}{}

\begin{itemize}
    \item Founded my own company whilst at university. To date, I've planned, designed and developed over 10 websites for clients.
\end{itemize}

\divider

\cvevent{Trainee Web Developer}{BBC, Visual Journalism}{July 2013 -- Aug 2014}{}

\begin{itemize}
    \item Developed the \emph{Commonwealth Games} feature, which was televised on BBC Breakfast. I also took the initiative to translate it to Welsh.
    \item Templated the ‘datapic’ format. Blog: \url{http://bit.ly/1wT6XTz}
\end{itemize}

\cvsection{Education}

\cvevent{B.Eng. in Software Engineering}{Aberystwyth University, Wales}{Sept 2011 -- June 2015}{}

\textbf{1st class, with Honours; BCS-accredited}

Technologies used: Java, C++, C, Haskell, Python, Ruby, PHP

Key modules: Advanced Computer Graphics, Agile Methodologies, Machine Learning, Computer Vision, Data Structures \& Algorithms

\divider

\cvevent{AAB (A-Levels), B (AS Level), 13 GCSEs (11 at A*/A)}{Lampeter Comprehensive School, Wales}{}{}

\cvsection{Publications}

\nocite{*}

\printbibliography[heading=pubtype,title={\printinfo{\faBook}{Newsletters}},type=newsletter]

\printbibliography[heading=pubtype,title={\printinfo{\faUsers}{Talks}},type=inproceedings]

%\divider

\printbibliography[heading=pubtype,title={\printinfo{\faBook}{Books}},type=book]

%\divider

\printbibliography[heading=pubtype,title={\printinfo{\faFile*[regular]}{Articles}}, type=misc]


%% Switch to the right column. This will now automatically move to the second
%% page if the content is too long.
\switchcolumn

\cvsection{Summary}
\begin{quote}
\textbf{Experienced full-stack web developer comfortable with a wide range of technologies, seeks senior back-end role.}
\\ \medskip
I am a creative, pragmatic and organised individual with experience of leading teams, delivering projects, mentoring and line management.

\end{quote}

\bigskip

Passions:
\cvtag{a11y}
\cvtag{perf}
\cvtag{standards}

\medskip

Values:
\hspace{2mm} % a bit of manual fudging to get labels to line up
\cvtag{diversity}
\cvtag{flexibility}
\cvtag{autonomy}

% deliberately no \medskip here as the tags above are a bit long and end up adding some margin

Enjoys:
\hspace{2mm} % a bit of manual fudging to get labels to line up
\cvtag{bouldering}
\cvtag{cooking}
\cvtag{singing}

\cvsection{Achievements}

\cvachievement{\faTrophy}{Graduated top of university class}{Awarded \emph{Best Bachelor Degree in Computer Science} out of approx 80 people}

\divider

\cvachievement{\faTrophy}{`Exceeded' rating}{Achieved highest rating in GDS end-of-year review, March 2020}

\divider

\cvachievement{\faTrophy}{Reduced GOV.UK errors by over 90\%}{Led a stream of work to reduce the high volumes of application errors across GOV.UK.}

\divider

\cvachievement{\faTrophy}{Conference speaker}{Spoke at Paris Web 2018}

\divider

\cvachievement{\faTrophy}{Smashing Magazine contributor}{Published a series of articles on \mbox{\href{https://www.smashingmagazine.com/author/chrisbashton/}{smashingmagazine.com}}}

\divider

\cvachievement{\faTrophy}{Run the \emph{frequent11y} newsletter}{To date, I've distributed almost 400 accessibility newsletters to hundreds of subscribers.}

\divider

\cvachievement{\faTrophy}{Best Contributor of the Year}{Awarded at the BBC News Developer Awards 2015}

\divider

\cvachievement{\faTrophy}{BBC Symphony Chorus}{Have sung in several televised Proms since joining in 2013}

\newpage

\cvsection{Projects}

\def\spacer{0.7cm}

\textbf{StudentMunch.com} - food review site, using structured data markup to show as reviews in search results. Built as a Progressive Web App on top of AMP.

\medskip

  \cvtag{Service Workers}
  \cvtag{AMP}
  \cvtag{Offline first}
  \\
  \cvtag{PWA}
  \cvtag{PRPL pattern}
  \cvtag{Structured data}

\vspace{\spacer}

\textbf{peepshowquot.es} - fan website for Channel 4 sitcom \emph{Peep Show}: a searchable directory of gifs for every quote.

\medskip

  \cvtag{React}
  \cvtag{NodeJS}
  \cvtag{Jamstack}
  \\
  \cvtag{AWS Lambda}
  \cvtag{Serverless}

\vspace{\spacer}

\textbf{SmartResolution.org} - online dispute resolution platform with plugin support for AI-driven resolutions. Awarded the Best Major Project prize at university.

\medskip

  \cvtag{Cucumber \& Ruby}
  \cvtag{PHP}
  \cvtag{Composer}
  \cvtag{SQL}
  \cvtag{PDO}
  \cvtag{LAMP}

\vspace{\spacer}

\textbf{Secretary} - automatic quality assurance editorial checks for WordPress posts. Built to be extensible and config-driven.
\newline
\href{https://wordpress.org/plugins/secretary/}{\url{wordpress.org/plugins/secretary/}}

\medskip

  \cvtag{WordPress}
  \cvtag{YAML}
  \cvtag{SVN}
  \cvtag{Open-source}

\medskip

\textbf{3D Solar System} - solar system written in WebGL.
\newline
\href{https://chrisbashton.github.io}{\url{chrisbashton.github.io}}

\medskip

  \cvtag{WebGL}
  \cvtag{JavaScript}
  \cvtag{HTML5}
  \cvtag{CSS3}

\medskip

%\cvsection{Languages}
%\cvskill{English}{5}
%\cvskill{Welsh}{3}
%\newpage

\cvsection{Referees}

\customreferences

\end{paracol}

\end{document}
